\documentclass[a4paper,12pt]{article}

\usepackage[BoldFont,SlantFont,CJKchecksingle]{xeCJK}
\usepackage[top=1.25in,bottom=1.25in,left=1in,right=1in]{geometry}
\usepackage{amsmath,amsthm,amssymb,latexsym,color}
\usepackage[CJKbookmarks=true,
            bookmarksnumbered=true,
            bookmarksopen=true,
            colorlinks =true, %注释掉此项则交叉引用为彩色边框
            pdfborder=001,    %注释掉此项则交叉引用为彩色边框
            citecolor =blue,%
            linkcolor =blue,
            ]{hyperref}
\usepackage{layout}
\usepackage{enumerate}
\usepackage{graphicx}
\usepackage{subfigure}

\newcommand{\mv}[1]{\mbox{\boldmath$#1$}}         % 自定义矩阵向量 斜+黑
\newcommand{\id}{\mathrm{d}}                      % 积分测度改为正体

\setCJKmainfont[BoldFont=黑体]{宋体}
\setCJKsansfont{STKaiti}% 设置缺省中文字体

\parindent 2em   %段首缩进

\begin{document}

\title{单物品拍卖机制设计}
\author{李林静 \\ 
   E-Mail: linjing.li@ia.ac.cn  \vspace{1cm} \\
 
 
        中国科学院自动化研究所 (CASIA)\\
        复杂系统管理与控制国家重点实验室 (SKLMCCS)\\
        互联网大数据与安全信息学研究中心 (iBasic)  \vspace{1cm} \\ 
 
 
        中国科学院大学人工智能学院 \vspace{5cm}
}

\date{Version: \today}



\maketitle



\newpage



 \section{机制设计}

 \subsection{背景}

设某卖者~(Seller)~决定以拍卖的方式出售一不可分物品~(indivisible
good),且该物品对卖者的保留价值~(Reserve
value)~$t_0$~为共同知识~(Common Knowledge)。
设有~$N$~个投标人参与竞价,其全体构成集合
\begin{equation*}
    \mathcal{N}=\{1,2,\cdots,N\}
\end{equation*}

以下设投标人~$i$~对标的物的估价~(Value)~为私有信息~(Private
Information),亦即其类型~(Type)。
在分析过程中令随机变量~$T_i$~表示投标人~$i$~的估价,且只有投标人~$i$~
知道自己估价的真实取值~$t_i$。 设随机变量~$T_i$~的取值范围为
\begin{equation*}
    \mathcal{T}_i=\left[\underline{\theta}_i,\overline{\theta}_i\right]
\end{equation*}
其中,$\underline{\theta}_i$~和~$\overline{\theta}_i$~分别是可能的估价下界和上界,
并记估价~$T_i$~的概率分布函数和概率密度函数分别为~$F_i(t_i)$~和~$f_i(t_i)$。

为方便分析,以下将各投标人的类型合并为下述类型~(估价)~向量~(随机向量)
\begin{equation*}
    \mv{T}=(T_1,T_2,\cdots,T_N)
\end{equation*}
其取值空间和一次具体实现值则分别记为
\begin{eqnarray*}
% \nonumber to remove numbering (before each equation)
  \mv{\mathcal{T}} &=& \mathcal{T}_1\times\mathcal{T}_2\times\cdots\times\mathcal{T}_N \\
  \mv{t} &=& (t_1,t_2,\cdots,t_N)
\end{eqnarray*}
按照不完全信息静态博弈的分析框架,
以下假设联合概率分布~$F(\mv{t})$~为共同知识,且记其概率密度函数为~$f(\mv{t})$。
同时,根据博弈分析的习惯,定义下述符号
 \begin{eqnarray*}
% \nonumber to remove numbering (before each equation)
  \mv{t}_{-i} &\triangleq& (t_1,t_2,\cdots,t_{i-1},t_{i+1},\cdots,t_N) \\
  \mv{T}_{-i} &\triangleq& (T_1,T_2,\cdots,T_{i-1},T_{i+1},\cdots,T_N)~ \\
  \mv{\mathcal{T}}_{-i} &\triangleq& \mathcal{T}_1\times\mathcal{T}_2\times\cdots\mathcal{T}_{i-1}
                            \times\mathcal{T}_{i+1}\times\cdots\times\mathcal{T}_N \\
  f_{-i}(\mv{t}_{-i}) &\triangleq& f(t_1,t_2,\cdots,t_{i-1},t_{i+1},\cdots,t_N)
\end{eqnarray*}
根据上述定义,可知符号~$\mv{t}_{-i}$,$\mv{T}_{-i}$,
$\mv{\mathcal{T}}_{-i}$~和~$f_{-i}(\mv{t}_{-i})$~分别表示
投标人~$i$~所面对的投标人的类型值,类型随机向量,类型取值空间和类型的概率密度函数。
且有~$(t_i,\mv{t}_{-i})=\mv{t}$,$(T_i,\mv{T}_{-i})=\mv{T}$,
以及~$(\mathcal{T}_i,\mv{\mathcal{T}}_{-i})=\mv{\mathcal{T}}$。


 \subsection{机制}

为获得标的物,投标人需要提交自己的报价,
在一般的机制~(Mechanism)~中则称其为消息~(Message)。
令~$\mathcal{W}_i$~表示投标人~$i$~的消息空间,其一个具体的消息则记为~$w_i$。
同时,记机制的消息空间及一个具体的消息向量实现为
\begin{eqnarray*}
% \nonumber to remove numbering (before each equation)
  \mv{\mathcal{W}} &=& \mathcal{W}_1\times\mathcal{W}_2
                       \times\cdots\times\mathcal{W}_N \\
  \mv{w} &=& (w_1,w_2,\cdots,w_N)
\end{eqnarray*}

当消息向量~$\mv{w}$~提交后,需要据其确定标的物的分配方案和价格。
对投标人~$i$,记其获得标的物的概率为~$\delta_i$,
则一个分配方案可以表示为
\begin{equation*}
    \Delta = (\delta_1,\delta_2,\cdots,\delta_N),
              ~~~~\forall i\in\mathcal{N},~0 \le \delta_i \le 1,
              ~\delta_1 + \delta_2 + \cdots + \delta_N \le 1
\end{equation*}
全体可能的分配方式则构成集合,
\begin{equation*}
    \mv{\Delta} = \left\{\Delta\ \left| \ \Delta = (\delta_1,\delta_2,\cdots,\delta_N),
                  ~~~~\forall i\in\mathcal{N},~0 \le \delta_i \le 1,
                  ~\delta_1 + \delta_2 + \cdots + \delta_N \le 1 \right.\right\}
\end{equation*}
即集合~$\mathcal{N}$~上的全体概率分布。注意到,当各分配概率之和小于~$1$~时,
存在卖者不出售``标的物''的可能性。为实现前述的根据消息向量来确定标的物的分配,需要预先指定一个
从消息空间~$\mv{\mathcal{W}}$~到集合~$\mv{\Delta}$~的映射
\begin{equation}\label{mech:map:allocation}
    \mv{\pi}:\mv{\mathcal{W}}\rightarrow\mv{\Delta}
\end{equation}
并称其为分配规则~(Allocation Rule)。
对任意消息向量~$\mv{w}\in\mv{\mathcal{W}}$,
$\pi_i(\mv{w})$~表示投标人~$i$~获得标的物的概率。

同样,为实现前述的根据消息向量来确定价格,
需要预先指定一个从消息空间~$\mv{\mathcal{W}}$~到价格空间~$\mathbb{R}^N$~的映射
\begin{equation}\label{mech:map:payment}
    \mv{\mu}:\mv{\mathcal{W}}\rightarrow\mathbb{R}^N
\end{equation}
并称其为定价规则~(Payment Rule),
对任意消息向量~$\mv{w}\in\mv{\mathcal{W}}$,
$\mu_i(\mv{w})$~表示投标人~$i$~所要支付的期望价格。

利用上述符号,(拍卖)机制~$\mathcal{M}$~可描述为如下三元组
\begin{equation}\label{mech:definition:mechanism}
    \mathcal{M} = \left(\mv{\mathcal{W}},\mv{\pi},\mv{\mu}\right)
\end{equation}
同时机制~$\mathcal{M}$~也定义了一个投标人之间的不完全信息静态博弈。
在该博弈中,投标人~$i$~的纯策略是从其类型空间~$\mathcal{T}_i$~到其
消息空间~$\mathcal{W}_i$~的映射
\begin{equation}\label{mech:map:pure_strategy}
    \beta_i:\mathcal{T}_i\rightarrow\mathcal{W}_i
\end{equation}
而策略组合则可表示为下述函数向量
\begin{equation*}
    \mv{\beta}=(\beta_1,\beta_2,\cdots,\beta_N)=(\beta_i,\mv{\beta}_{-i})
\end{equation*}
其中,$\mv{\beta}_{-i}$~为投标人~$i$~所面临的对手策略
\begin{equation*}
    \mv{\beta}_{-i} \triangleq (\beta_1,\beta_2,\cdots,
                                  \beta_{i-1},\beta_{i+1},\cdots,\beta_N)
\end{equation*}
按照定义,策略组合~$\mv{\beta}$~构成为博弈
(或机制)~$\mathcal{M}$~的~Bayesian Nash~均衡~(Bayesian Nash
Equilibrium,BNE),若~$\forall i\in\mathcal{N}$,
如果除~$i$~之外的投标人使用策略~$\mv{\beta}_{-i}$,
则采用~$\beta_i$~是投标人~$i$~的最优策略。

 \subsection{直接机制与显示原理}\label{sec:survey:direct:mechanism}

在设计机制的时候,可选择的消息类型可以无限多,这为求解制造了障碍,
特别是在求解最优机制的时候。若一机制的消息空间跟投标人的类型空间完全相同,
即~$\forall
i\in\mathcal{N},~\mathcal{W}_i=\mathcal{T}_i$,则称其为直接机制~(Direct
Mechanism),并可用如下二元组表示
 \begin{equation}\label{mech:definition:direct:mechanism}
    \mathcal{D} = (\mv{\pi},\mv{\mu})
\end{equation}
此时,分配规则~$\mv{\pi}$~和定价规则~$\mv{\mu}$~定义在类型空间~$\mv{\mathcal{T}}$~上
\begin{eqnarray}
% \nonumber to remove numbering (before each equation)
  \mv{\pi}&:&\mv{\mathcal{T}}\rightarrow\mv{\Delta} \\
  \mv{\mu}&:&\mv{\mathcal{T}}\rightarrow\mathbb{R}^N
\end{eqnarray}

对一个直接机制~$\mathcal{D}$,如果说实话构成一个~BNE~($\forall
i,~\forall
t_i\in\mathcal{T}_i,~\beta_i(t_i)=t_i$)~,则称该直接机制是说实话的~(Truthful)。\vspace{5pt}

\textbf{显示原理~(Revelation Principle)~~~~
对任意机制~$\mathcal{M}$,若其存在均衡~$\mv{\beta}$,
则在该均衡上的配置结果~(分配和价格),可以通过一说实话的直接机制来实现。}\vspace{5pt}

对任意机制~$\mathcal{M}=(\mv{\mathcal{W}},\mv{\pi},\mv{\mu})$~及其均衡~$\mv{\beta}$,
能实现上述功能的直接机制~$\mathcal{D}^\mathcal{M}=(\mv{\pi}^\mathcal{M},\mv{\mu}^\mathcal{M})$~可取
为~(\mv{\pi},\mv{\mu})~和~$\mv{\beta}$~的复合
 \begin{eqnarray*}
% \nonumber to remove numbering (before each equation)
  \mv{\pi}^\mathcal{M} &=& \mv{\pi}(\mv{\beta})=\mv{\pi}\circ\mv{\beta} \\
  \mv{\mu}^\mathcal{M} &=& \mv{\mu}(\mv{\beta})=\mv{\mu}\circ\mv{\beta}
\end{eqnarray*}
注意到,当投标人说实话时,将类型(估价)~$\mv{t}$~带入直接机制~$\mathcal{D}^\mathcal{M}$,
的确可得到跟原机制~$\mathcal{M}=(\mv{\mathcal{W}},\mv{\pi},\mv{\mu})$~在
均衡~$\mv{\beta}$~上相同的配置。
故只需要证明说实话是机制~$\mathcal{D}^\mathcal{M}$~的均衡。
为此,假设除~$i$~之外的投标人均说实话,
而投标人~$i$~的报价~$s_i \neq t_i$,其期望收益为
\begin{eqnarray*}
% \nonumber to remove numbering (before each equation)
  \pi^\mathcal{M}_i(s_i,\mv{t}_{-i})t_i - \mu_i^\mathcal{M}(s_i,\mv{t}_{-i}) &=&
   \pi_i(\mv{\beta}(s_i,\mv{t}_{-i}))t_i - \mu_i(\mv{\beta}(s_i,\mv{t}_{-i}))\\
   &=& \pi_i(\beta(s_i),\mv{\beta}_{-i}(\mv{t}_{-i}))t_i - \mu_i(\beta(s_i),\mv{\beta}_{-i}(\mv{t}_{-i}))\\
   &\leq& \pi_i(\beta(t_i),\mv{\beta}_{-i}(\mv{t}_{-i}))t_i - \mu_i(\beta(t_i),\mv{\beta}_{-i}(\mv{t}_{-i}))\\
   &=& \pi_i(\mv{\beta}(\mv{t}))t_i - \mu_i(\mv{\beta}(\mv{t}))\\
   &=& \pi^\mathcal{M}_i(\mv{t})t_i - \mu_i^\mathcal{M}(\mv{t})
\end{eqnarray*}
因为~$\mv{\beta}$~是均衡,故不等号成立。由上述各式可知当对手使用说实话策略时,
任意投标人~$i$~的最佳策略也是说实话,故说实话是直接机制~$\mathcal{D}^\mathcal{M}$~的均衡。

根据显示原理,在设计机制的时候,
可以直接将投标人的消息空间~$\mv{\mathcal{W}}$~取为对应的类型空间~$\mv{\mathcal{T}}$,
从而降低设计的复杂度。


 \subsection{激励兼容与收益等价}\label{sec:survey:incentive:IC}

对直接机制~$\mathcal{D}=(\mv{\pi},\mv{\mu})$,若投标人~$i$~报价~$s_i$,
而其对手均采用说实话策略,则投标人~$i$~能获得``标的物''的期望概率和
需要支付的期望价格分别为
 \begin{eqnarray}
\label{mech:definition:expected:prob}
\delta_i(s_i) &=& \mathbb{E}[\pi_i(s_i, \mv{T}_{-i})] = \int_{\mv{\mathcal{T}}_{-i}}
    \pi_i(s_i,\mv{t}_{-i})f_{-i}(\mv{t}_{-i})~\id\mv{t}_{-i}\\
\label{mech:definition:expected:payment}
  m_i(s_i) &=& \mathbb{E}[\mu_i(s_i, \mv{T}_{-i})] = \int_{\mv{\mathcal{T}}_{-i}}
    \mu_i(s_i,\mv{t}_{-i})f_{-i}(\mv{t}_{-i})~\id\mv{t}_{-i}
\end{eqnarray}
注意上述概率和价格只跟投标人~$i$~的报价相关联,而与其真实估价无关。
当投标人~$i$~的真实估价为~$t_i$~时,其能获得的期望收益为
\begin{eqnarray}
    \nonumber
    u_i(t_i,s_i) &=& \mathbb{E}[\pi_i(s_i, \mv{T}_{-i})t_i - \mu_i(s_i, \mv{T}_{-i})] 
                  =  \mathbb{E}[\pi_i(s_i, \mv{T}_{-i})]t_i - \mathbb{E}[\mu_i(s_i, \mv{T}_{-i})] \\
                 &=& \delta_i(s_i)t_i - m_i(s_i)
		 \label{mech:definition:expected:payoff}
\end{eqnarray}

称直接机制~$\mathcal{D}$~为激励兼容的~(Incentive
Compatible,IC),
若~$\forall i\in\mathcal{N}, ~\forall t_i\neq s_i\in\mathcal{T}_i$,
下述关于期望收益的不等式成立
\begin{equation}\label{mech:definition:ic:inequality}
    U_i(t_i) \triangleq u_i(t_i,t_i) = \delta_i(t_i)t_i - m_i(t_i)
                \ge \delta_i(s_i)t_i - m_i(s_i) = u_i(t_i,s_i)
\end{equation}

由上述定义可知,若机制~$\mathcal{D}$~为激励兼容的,
则所有投标人使用说实话策略将构成机制的一个均衡~(BNE)。
故激励兼容也可如下定义
\begin{equation}\label{mech:definition:ic:max}
    U_i(t_i) \triangleq \max_{s_i\in\,\mathcal{T}_i}
                 ~\left\{\delta_i(s_i)t_i - m_i(s_i)\right\},
                 ~~~~\forall i\in\,\mathcal{N},~\forall t_i\in\mathcal{T}_i
\end{equation}

注意到在上述定义中,二元函数~$u_i(x,y)$~的第一变元~$x$~表示投标人~$i$~的类型,
第二变元~$y$~为其报价,$u_i(x,y)$~等于投标人~$i$~能获得的期望收益。
函数~$U_i(\cdot)$~则等于投标人~$i$~说实话时能获得的期望收益,
也称为均衡收益函数~(Equilibrium Payoff Function,因为说实话构成~BNE)。
以下分析激励兼容的直接机制所具有的特性。\vspace{5pt}

\textbf{(1)~~对激励兼容的直接机制~$\mathcal{D}$,$\forall i\in\,\mathcal{N}$,
均衡收益函数~$U_i(\cdot)$~为凸函数。}\vspace{5pt}

因为对~$\forall t_i^1\neq t_i^2\in\,\mathcal{T}_i$,
$\forall \lambda_1,\lambda_2\ge 0, \lambda_1+\lambda_2=1$,成立
\begin{eqnarray*}
% \nonumber to remove numbering (before each equation)
  U_i\left(\lambda_1t_i^1 + \lambda_2t_i^2\right) &=& \max_{s_i\in\,\mathcal{T}_i}
     ~\left\{\delta_i(s_i)\left(\lambda_1t_i^1 + \lambda_2t_i^2\right) - m_i(s_i)\right\}\\
   &=& \max_{s_i\in\,\mathcal{T}_i}
     ~\left\{\lambda_1\delta_i(s_i)t_i^1 + \lambda_2\delta_i(s_i)t_i^2
     - (\lambda_1 + \lambda_2)m_i(s_i)\right\} \\
   &=& \max_{s_i\in\,\mathcal{T}_i}
     ~\left\{\lambda_1\left(\delta_i(s_i)t_i^1 - m_i(s_i)\right)
     + \lambda_2\left(\delta_i(s_i)t_i^2 - m_i(s_i)\right)\right\} \\
   &\leq& \max_{s_i\in\,\mathcal{T}_i}
     ~\left\{\lambda_1\left(\delta_i(s_i)t_i^1 - m_i(s_i)\right)\right\}
     + \max_{s_i\in\,\mathcal{T}_i}
     ~\left\{\lambda_2\left(\delta_i(s_i)t_i^2 - m_i(s_i)\right)\right\} \\
   &=& \lambda_1\max_{s_i\in\,\mathcal{T}_i}
     ~\left\{\delta_i(s_i)t_i^1 - m_i(s_i)\right\}
     + \lambda_2\max_{s_i\in\,\mathcal{T}_i}
     ~\left\{\delta_i(s_i)t_i^2 - m_i(s_i)\right\}\\
   &=& \lambda_1U_i\left(t_i^1\right) + \lambda_2U_i\left(t_i^2\right)
\end{eqnarray*}
式中不等号成立,因为任意两个函数之和的最大值不可能超过两个函数的最大值之和,
即有~$\max\{f+g\}\le\max~f+\max~g$。按凸函数的定义,
可得均衡收益函数~$U_i(\cdot)$~为凸函数~(注意到函数~$U_i(\cdot)$~定义
在区间~$\mathcal{T}_i$~上,而区间~$\mathcal{T}_i$~为凸集)。
因为凸函数均为绝对连续函数,故~$U_i(\cdot)$~在定义域内部几乎处处可导。

\textbf{(2)~~对激励兼容的直接机制~$\mathcal{D}$,
均衡收益函数~$U_i(\cdot)$~在定义域内部几乎处处可导,
且在可微点上有~$U'_i(t_i) = \delta_i(t_i)$。}\vspace{5pt}

注意到,对~$\forall t_i\neq s_i\in\mathcal{T}_i$,以下不等式组成立
\begin{eqnarray*}
% \nonumber to remove numbering (before each equation)
  U_i(t_i) &\ge& u_i(t_i,s_i) = \delta_i(s_i)t_i - m_i(s_i) \\
   &=& \delta_i(s_i)s_i - m_i(s_i) + \delta_i(s_i)t_i - \delta_i(s_i)s_i\\
   &=& U_i(s_i) + \delta_i(s_i)(t_i - s_i)\\
  U_i(s_i) &\ge& u_i(s_i,t_i) = \delta_i(t_i)s_i - m_i(t_i) \\
   &=& \delta_i(t_i)t_i - m_i(t_i) + \delta_i(t_i)s_i - \delta_i(t_i)t_i\\
   &=& U_i(t_i) + \delta_i(t_i)(s_i - t_i)
\end{eqnarray*}
若有~$t_i>s_i$,则由上述不等式组可得
\begin{eqnarray*}
% \nonumber to remove numbering (before each equation)
  \delta_i(s_i) &\le& \frac{U_i(t_i) - U_i(s_i)}{t_i - s_i} \\
  \delta_i(t_i) &\ge& \frac{U_i(t_i) - U_i(s_i)}{t_i - s_i}
\end{eqnarray*}
由~$\delta_i(\cdot)$~的定义式~(\ref{mech:definition:expected:prob})~可知其为连续函数,
联合极限的保序性可得
\begin{eqnarray*}
% \nonumber to remove numbering (before each equation)
  \delta_i(t_i) = \lim_{s_i\rightarrow t_i^-}\delta_i(s_i)
              &\le& \lim_{s_i\rightarrow t_i^-}\frac{U_i(t_i) - U_i(s_i)}{t_i - s_i}
                = U'_i(t_i^-)\\
  \delta_i(t_i) &\ge& \lim_{s_i\rightarrow t_i^-}\frac{U_i(t_i) - U_i(s_i)}{t_i - s_i}
                  = U'_i(t_i^-)
\end{eqnarray*}
联合上述不等式组可得
\begin{equation*}
    U'_i(t_i^-) \le \delta_i(t_i) \le U'_i(t_i^-)
\end{equation*}
若有~$t_i<s_i$,则类似的分析可得
\begin{equation*}
    U'_i(t_i^+) \le \delta_i(t_i) \le U'_i(t_i^+)
\end{equation*}
在函数的可微点上,其左导数跟右导数相等,
故均衡收益函数~$U_i(\cdot)$~的可微点上有~$U'_i(t_i) = \delta_i(t_i)$。
对于几乎处处可导,前面已经由其绝对连续性说明。
于是均衡收益函数~$U_i(\cdot)$~可以表示为下述积分形式
\begin{equation}\label{mech:rep:epf}
    U_i(t_i) = U_i(\underline{\theta}_i) + \int_{\underline{\theta}_i}^{t_i}~\delta_i(x)\id x
\end{equation}

\textbf{(3)~~对直接机制~$\mathcal{D}$,
激励兼容性等价于概率函数~$\delta_i(\cdot)$~单调不减,$\forall i\in\,\mathcal{N}$。}\vspace{5pt}

首先,在前一段分析中已经得到,若机制~$\mathcal{D}$~激励兼容,则当~$t_i>s_i$~时,成立
\begin{equation*}
    \delta_i(s_i) \le \frac{U_i(t_i) - U_i(s_i)}{t_i - s_i} \le \delta_i(t_i)
\end{equation*}
即由机制的激励兼容性可以得到概率函数~$\delta_i(\cdot)$~单调不减。
反之,若概率函数~$\delta_i(\cdot)$~单调不减。
则对~$\forall t_i\neq s_i\in\mathcal{T}_i$,
由均衡收益函数~$U_i(\cdot)$~的积分表示~(\ref{mech:rep:epf})~可得

\begin{eqnarray*}
% \nonumber to remove numbering (before each equation)
  U_i(t_i) - U_i(s_i) &=& \int_{s_i}^{t_i}~\delta_i(x)\id x
   \ge \delta_i(s_i)(t_i - s_i)\\
   &\Downarrow&\\
  U_i(t_i) &\ge& U_i(s_i) + \delta_i(s_i)(t_i - s_i)\\
     &=& \delta_i(s_i)s_i - m_i(s_i) + \delta_i(s_i)(t_i - s_i)\\
     &=& \delta_i(s_i)t_i - m_i(s_i)\\
     &=& u_i(t_i,s_i)
\end{eqnarray*}
概率函数~$\delta_i(\cdot)$~的单调不减性保证式中不等号成立。
因为利用积分中值定理,可得
\begin{equation*}
    \int_{s_i}^{t_i}~\delta_i(x)\id x = \left\{
        \begin{array}{ll}
          \delta_i(\xi)(t_i - s_i), & s_i\le\xi\le t_i,~\mathrm{if}~t_i > s_i \\
          \delta_i(\xi)(t_i - s_i), & t_i\le\xi\le s_i,~\mathrm{if}~t_i < s_i
        \end{array}
    \right.
\end{equation*}
对~$t_i > s_i$~的情况,由函数~$\delta_i(\cdot)$~单调不减,
可得~$\delta_i(\xi)\ge\delta_i(s_i)$,从而不等号成立,即
\begin{equation*}
    \int_{s_i}^{t_i}~\delta_i(x)\id x = \delta_i(\xi)(t_i - s_i)
                                     \ge \delta_i(s_i)(t_i - s_i)
\end{equation*}
对~$t_i < s_i$~的情况,由函数~$\delta_i(\cdot)$~单调不减,
可得~$\delta_i(\xi)\le\delta_i(s_i)$,因为~$t_i - s_i$~为负,
故
\begin{equation*}
    \delta_i(\xi)(t_i - s_i)\ge\delta_i(s_i)(t_i - s_i)
\end{equation*}
即前述不等式依然成立。\vspace{5pt}

\textbf{(4)~~收益等价定理~(Revenue Equivalence
Theorem,RET)。}\vspace{5pt}

联合式~(\ref{mech:definition:expected:prob})~和~(\ref{mech:rep:epf})~可得
\begin{eqnarray}
\nonumber
  U_i(t_i) &=& U_i(\underline{\theta}_i) + \int_{\underline{\theta}_i}^{t_i}~\delta_i(x)\id x \\
\nonumber
   &=& U_i(\underline{\theta}_i) + \int_{\underline{\theta}_i}^{t_i}~
       \int_{\mv{\mathcal{T}}_{-i}}\pi_i(x,\mv{t}_{-i})f_{-i}(\mv{t}_{-i})~\id\mv{t}_{-i} \id x\\
   &=& U_i(\underline{\theta}_i) + \int_{\mv{\mathcal{T}}_{-i}}~
       \left(\int_{\underline{\theta}_i}^{t_i}\pi_i(x,\mv{t}_{-i})~\id x\right)~f_{-i}(\mv{t}_{-i})~\id\mv{t}_{-i}
\label{mech:ret:epf}
\end{eqnarray}
注意到式中的积分项只依赖于机制的分配规则和投标人的类型分布。因此,
若两个激励兼容的直接机制采用相同的分配规则,
则投标人~$i$~在这两个机制下的均衡收益在相差常数的意义下等价,
且该常数为投标人~$i$~在类型为~$\underline{\theta}_i$~时的均衡收益~$U_i(\underline{\theta}_i)$。
如果上述最低均衡收益相等,则投标人~$i$~在这两种机制下的均衡收益相等。
利用均衡收益和期望价格间的关系~(\ref{mech:definition:ic:inequality})~可得
均衡时投标人~$i$~的期望价格为
\begin{eqnarray*}
% \nonumber to remove numbering (before each equation)
  m_i(t_i) &=& \delta_i(t_i)t_i - U_i(t_i) \\
   &=& -U_i(\underline{\theta}_i) + \delta_i(t_i)t_i - \int_{\underline{\theta}_i}^{t_i}~\delta_i(x)\id x \\
   &=& m_i(\underline{\theta}_i) + \delta_i(t_i)t_i - \delta_i(\underline{\theta}_i)\underline{\theta}_i
    - \int_{\underline{\theta}_i}^{t_i}~\delta_i(x)\id x
\end{eqnarray*}
注意到对最低估价~$\underline{\theta}_i$,均衡收益为
\begin{equation*}
    U_i(\underline{\theta}_i) = \delta_i(\underline{\theta}_i)\underline{\theta}_i - m_i(\underline{\theta}_i)
\end{equation*}
故也可以说若两个激励兼容的直接机制采用相同的分配规则,
则投标人~$i$~在这两个机制下的期望价格在相差常数的意义下等价,
且该常数为投标人~$i$~在类型为~$\underline{\theta}_i$~时的期望价格~$m_i(\underline{\theta}_i)$,
若上述最低期望价格相同,则~$i$~在两种机制下的期望价格相等。

对卖者而言,其期望收益则可表示为
\begin{eqnarray*}
  U_0 &=& \int_{\mv{\mathcal{T}}}\left[t_0\left(1-\sum_{i\in\,\mathcal{N}}\pi_i(\mv{t})\right)
       +  \sum_{i\in\,\mathcal{N}}\mu_i(\mv{t})\right]f(\mv{t})\id\mv{t} \\
      &=& t_0-t_0\sum_{i\in\,\mathcal{N}}\int_{\mathcal{T}_i}~\delta_i(t_i)f_i(t_i)\id t_i
       +  \sum_{i\in\,\mathcal{N}}\int_{\mathcal{T}_i}~m_i(t_i)f_i(t_i)\id t_i\\
      &=& t_0+\sum_{i\in\,\mathcal{N}}\int_{\mathcal{T}_i}~\left(-t_0\delta_i(t_i)
          -U_i(\underline{\theta}_i) + \delta_i(t_i)t_i -
          \int_{\underline{\theta}_i}^{t_i}~\delta_i(x)\id x\right)f_i(t_i)\id t_i\\
      &=& t_0 + \sum_{i\in\,\mathcal{N}}\int_{\mathcal{T}_i}~
          \left(\delta_i(t_i)(t_i-t_0) - \int_{\underline{\theta}_i}^{t_i}~\delta_i(x)\id x\right)f_i(t_i)\id t_i
       - \sum_{i\in\,\mathcal{N}}U_i(\underline{\theta}_i)
\end{eqnarray*}
注意到上式中只有最后的求和项依赖于机制的定价规则。因此,
若两个激励兼容的直接机制采用相同的分配规则,
则卖者在这两个机制下的期望收益在相差常数的意义下等价,
且该常数为各投标人在类型为~$\underline{\theta}_i$~时的
期望价格之和~$\sum_{i\in\,\mathcal{N}}U_i(\underline{\theta}_i)$。



\subsection{个体理性}

卖者不可能强迫投标人来买其提供的``标的物'',故当购得``标的物''时,
投标人要能够获得正的收益,即要求直接机制~$\mathcal{D}=(\mv{\pi},\mv{\mu})$~是
个体理性的~(Individual Rational,IR),亦即满足下述条件
\begin{equation}\label{mech:IR:definition}
    U_i(t_i) \ge 0,~~~~\forall i\in\,\mathcal{N},~~t_i\in\mathcal{T}_i
\end{equation}
注意到当机制~$\mathcal{D}$~激励兼容时,由式~(\ref{mech:rep:epf})~和~$\delta_i(\cdot)$~的
非负性可得~IR~条件等价于
\begin{equation*}
    U_i(\underline{\theta}_i) \ge 0,~~~~\forall i\in\,\mathcal{N}
\end{equation*}



\section{最优机制}

称同时满足~IC~和~IR~两个条件,
且能最大化卖者收益的直接机制~$\mathcal{D}^\star=(\mv{\pi}^\star,\mv{\mu}^\star)$~为最优机制~(OPT)。
以下讨论一般意义下最优机制的求解问题,分析中假定~$f_i(t)>0$,$\forall
t\in\mathcal{T}_i$,且各投标人的类型~(估价)~分布独立,即有
\begin{equation}
    f(\mv{t}) = \prod_{i\in\,\mathcal{N}}f_i(t_i)
\end{equation}

\subsection{最优机制求解}

利用分步积分法,可得
\begin{eqnarray*}
% \nonumber to remove numbering (before each equation)
  \int_{\mathcal{T}_i}\int_{\underline{\theta}_i}^{t_i}~\delta_i(x)\id x f_i(t_i)\id t_i &=&
  \int_{\mathcal{T}_i}\int_{\underline{\theta}_i}^{t_i}~\delta_i(x)\id x \id F_i(t_i) \\
   &=& \left.F_i(t_i)\int_{\underline{\theta}_i}^{t_i}~\delta_i(x)\id x
       \right|_{\underline{\theta}_i}^{\overline{\theta}_i}
    -  \int_{\mathcal{T}_i}F_i(t_i)\delta_i(t_i)\id t_i\\
   &=& \int_{\mathcal{T}_i}\delta_i(t_i)\id t_i - \int_{\mathcal{T}_i}F_i(t_i)\delta_i(t_i)\id t_i \\
   &=& \int_{\mathcal{T}_i}\left(1 - F_i(t_i)\right)\delta_i(t_i)\id t_i
\end{eqnarray*}
故卖者的期望收益可表示为
\begin{eqnarray}
\nonumber
  U_0 &=& \sum_{i\in\,\mathcal{N}}\int_{\mathcal{T}_i}~
          \left(t_i -\frac{1 - F_i(t_i)}{f_i(t_i)} - t_0\right)\delta_i(t_i)f_i(t_i)\id t_i
          + t_0 - \sum_{i\in\,\mathcal{N}}U_i(\underline{\theta}_i) \\
\nonumber
      &=& \sum_{i\in\,\mathcal{N}}\int_{\mathcal{T}_i}~
          \left(t_i -\frac{1 - F_i(t_i)}{f_i(t_i)} - t_0\right)\int_{\mv{\mathcal{T}}_{-i}}
          \pi_i(t_i,\mv{t}_{-i})f_{-i}(\mv{t}_{-i})~\id\mv{t}_{-i}f_i(t_i)\id t_i
          + t_0 - \sum_{i\in\,\mathcal{N}}U_i(\underline{\theta}_i)\\
      &=& \int_{\mv{\mathcal{T}}}\left[\sum_{i\in\,\mathcal{N}}~
          \left(t_i -\frac{1 - F_i(t_i)}{f_i(t_i)} - t_0\right)\pi_i(\mv{t})\right]f(\mv{t})~\id\mv{t}
          + t_0 - \sum_{i\in\,\mathcal{N}}U_i(\underline{\theta}_i)
\label{mech:revenue:seller}
\end{eqnarray}
为最大化上式,应当使得积分项最大化,同时使最后的求和项最小化。
注意到积分项中的决策变量只包括分配规则,故最优机制~$\mathcal{D}^\star$~的
分配规则~$\mv{\pi}^\star$~满足条件
\begin{equation}\label{mech:optimal:allocation}
\begin{array}{rl}
 \mv{\pi}^\star = \mathop{\arg\max}\limits_{\mv{\pi}} & \int_{\mv{\mathcal{T}}}\left[\sum\limits_{i\in\,\mathcal{N}}~
                       \left(t_i -\frac{1 - F_i(t_i)}{f_i(t_i)} - t_0\right)\pi_i(\mv{t})\right]f(\mv{t})~\id\mv{t} \\
                       \mathrm{s.t.} & \sum\limits_{i\in\,\mathcal{N}}\pi_i(\mv{t})\le 1 \\
                        & \pi_i(\mv{t})\ge 0,~~\forall i\in\,\mathcal{N}
                     \end{array}
\end{equation}
得到最优分配规则后,再选择定价规则使得求和项最小。注意到满足~IC~和~IR~条件的机制,
其~$U_i(\underline{\theta}_i) \ge 0$,故应当选择定价规则,使得
\begin{equation*}
    U_i(\underline{\theta}_i) = 0,~~~~\forall i\in\,\mathcal{N}
\end{equation*}
由式~(\ref{mech:ret:epf})~可得
\begin{eqnarray*}
% \nonumber to remove numbering (before each equation)
  U_i(\underline{\theta}_i) &=& U_i(t_i) - \int_{\mv{\mathcal{T}}_{-i}}~
       \left(\int_{\underline{\theta}_i}^{t_i}\pi_i(x,\mv{t}_{-i})~\id x\right)~f_{-i}(\mv{t}_{-i})~\id\mv{t}_{-i}\\
   &=& \delta_i(t_i)t_i - m_i(t_i) - \int_{\mv{\mathcal{T}}_{-i}}~
       \left(\int_{\underline{\theta}_i}^{t_i}\pi_i(x,\mv{t}_{-i})~\id x\right)~f_{-i}(\mv{t}_{-i})~\id\mv{t}_{-i}\\
   &=& \int_{\mv{\mathcal{T}}_{-i}}t_i\pi_i(t_i,\mv{t}_{-i})f_{-i}(\mv{t}_{-i})~\id\mv{t}_{-i}
        - m_i(t_i) - \int_{\mv{\mathcal{T}}_{-i}}~
       \left(\int_{\underline{\theta}_i}^{t_i}\pi_i(x,\mv{t}_{-i})~\id x\right)~f_{-i}(\mv{t}_{-i})~\id\mv{t}_{-i}\\
\end{eqnarray*}
令~$U_i(\underline{\theta}_i) = 0$~可得到
\begin{equation*}
    m_i(t_i) = \int_{\mv{\mathcal{T}}_{-i}}\left[t_i\pi_i(\mv{t})
      - \int_{\underline{\theta}_i}^{t_i}\pi_i(x,\mv{t}_{-i})~\id x\right] f_{-i}(\mv{t}_{-i})~\id\mv{t}_{-i}
\end{equation*}
再由~$m_i(t_i)$~的定义式~(\ref{mech:definition:expected:payment})~可得最优机制中的定价规则为
\begin{equation}\label{mech:optimal:payment}
    \mu_i^\star(\mv{t}) = t_i\pi_i(\mv{t}) - \int_{\underline{\theta}_i}^{t_i}\pi_i(x,\mv{t}_{-i})~\id x,
    ~~~~\forall i\in\,\mathcal{N}
\end{equation}


\subsection{正规最优机制}

定义投标人~$i$~的实质估价~(virtual valuation)~为
\begin{equation}\label{mech:definition:virtual:valuation}
    \psi_i(t_i) = t_i - \frac{1 - F_i(t_i)}{f_i(t_i)} = t_i - \frac{1}{\lambda_i(t_i)}
\end{equation}
式中,$\lambda_i(\cdot)$~为失效率~(Hazard Rate)。称最优机制设计问题为\textbf{正规的~(Regular)},
若全体投标人的实质估价函数均单调不减。\vspace{5pt}

\noindent\textbf{正规条件下的最优机制
\begin{enumerate}[(1)]
  \item 若~$\forall i\in\,\mathcal{N},~\psi_i(t_i) < t_0$,则卖者保留``标的物'';
  \item 若~$\exists i\in\,\mathcal{N}$,使得~$\psi_i(t_i) \ge t_0$,
        则将``标的物''分配给实质估价最高的投标人,若有多个投标人的实质估价相同,则以相同的概率
        随机分配``标的物'';
  \item 定价规则按式~(\ref{mech:optimal:payment})~计算。\vspace{5pt}
\end{enumerate}
}

首先,上面给出的机制最大化卖者的收益,并满足~IR~条件,以下说明其满足~IC~条件。
注意到~IC~条件等价于~$\delta_i(\cdot)$~单调不减。$\forall s_i < t_i$,当实质估价单调不减时,
$\psi_i(s_i) < \psi_i(t_i)$,故~$\pi_i(s_i,\mv{t}_{-i}) < \pi_i(t_i,\mv{t}_{-i})$,
由~$\delta_i(\cdot)$~的定义式知~$\delta_i(s_i)<\delta_i(t_i)$,故有~$\delta_i(\cdot)$~单调不减。

为方便求解价格,定义
\begin{equation}\label{mech:payment:intuitive}
    z_i(\mv{t}_{-i}) \triangleq \inf\left\{x\left|\psi_i(x) \ge t_0,~\psi_i(x)
                     \ge \psi_j(t_j),~\forall j\neq i\right.\right\}
\end{equation}
即~$z_i(\mv{t}_{-i})$~是当对手类型为~$\mv{t}_{-i}$~时,投标人~$i$~能赢得``标的物''的最低报价。
利用~$z_i(\mv{t}_{-i})$~可得分配规则的取值为
\begin{equation*}
    \pi_i(x,\mv{t}_{-i}) = \left\{
        \begin{array}{cl}
          1, & x > z_i(\mv{t}_{-i}) \\
          0, & x < z_i(\mv{t}_{-i})
        \end{array}
    \right.
\end{equation*}
同时可得到
\begin{equation*}
    \int_{\underline{\theta}_i}^{t_i}\pi_i(x,\mv{t}_{-i})~\id x = \left\{
        \begin{array}{cl}
          \int_{z_i(\textbf{\emph{t}}_{-i})}^{t_i}~1~\id x
             = t_i-z_i(\mv{t}_{-i}), & t_i > z_i(\mv{t}_{-i}) \\
          0, & t_i < z_i(\textbf{\emph{t}}_{-i})
        \end{array}
    \right.
\end{equation*}
从而得到最优的定价规则
\begin{equation}\label{mech:definition:optimal:payment}
    \mu_i^\star(\mv{t}) = t_i\pi_i(\mv{t}) - \int_{\underline{\theta}_i}^{t_i}\pi_i(x,\mv{t}_{-i})~\id x
                        = \left\{
        \begin{array}{cl}
          z_i(\mv{t}_{-i}), & \pi_i(\mv{t}) = 1 \\
          0, & \pi_i(\mv{t}) = 0
        \end{array}
    \right.
\end{equation}
于是正规条件下的最优机制可描述为
\begin{eqnarray*}
% \nonumber to remove numbering (before each equation)
  \pi_i^\star(\mv{t}) &=& \left\{
        \begin{array}{cl}
          1, & \psi_i(t_i) \ge t_0,~\psi_i(t_i) > \psi_j(t_j),~\forall j\neq i \\
          0, & \mathrm{otherwise}
        \end{array}
  \right. \\
  \mu_i^\star(\mv{t}) &=& \left\{
        \begin{array}{cl}
          z_i(\mv{t}_{-i}), & \pi_i(\mv{t}) = 1 \\
          0, & \pi_i(\mv{t}) = 0
        \end{array}
  \right.
\end{eqnarray*}

在对称情况下,各投标人具有相同的分布和实质估价函数,记为~$\psi(\cdot)$,可得
\begin{eqnarray*}
% \nonumber to remove numbering (before each equation)
  z_i(\mv{t}_{-i}) &=& \inf\left\{x\left|\psi(x) \ge t_0,~\psi(x)
                     \ge \psi(t_j),~\forall j\neq i\right.\right\} \\
   &=& \inf\left\{x\left|\psi(x) \ge t_0,~x
                     \ge t_j,~\forall j\neq i\right.\right\} \\
   &=& \max\left\{\psi^{-1}(t_0),~\max\limits_{j\neq i} \left\{t_j\right\}\right\}
\end{eqnarray*}
即对称情况下的最优机制为附带保留价~(Reserve Price)~$r^\star = \psi^{-1}(t_0)$~的二价机制。

\subsection{经济学含义}

设买者的估价分布为~$F(t)$,若卖者给出价格~$p$,则买者购买的概率为~$1-F(p)$,
即估价高于~$p$~的概率。若将这一概率视为买者的需求函数~(Demand Function),则有
\begin{equation*}
    q(p)=1-F(p)
\end{equation*}
反之有逆需求函数~(Inverse Demand Fucntion)
\begin{equation*}
    p=F^{-1}(1-q)
\end{equation*}
则卖者的期望收益为
\begin{equation*}
    R(q)=pq=F^{-1}(1-q)q
\end{equation*}
对上式微分,得到边际收益
\begin{eqnarray*}
% \nonumber to remove numbering (before each equation)
  MR &=& \frac{\id R}{\id q} = F^{-1}(1-q) - \frac{q}{f(p)|_{p=F^{-1}(1-q)}}
  \\
     &=& p - \frac{1 - F(p)}{f(p)}
\end{eqnarray*}
即实质估价为价格为~$p$~时卖者的边际收益,
而最优机制则按照各投标人为卖者提供的边际收益大小分配``标的物'',
价格则确定为该投标人能获得``标的物''的最低报价。

对于得到``标的物''的投标人,由实质估价~$\psi_i(\cdot)$~和~$z_i(\mv{t}_{-i})$~的定义可知,
其支付的价格不超过自己的估价,故其能获得的交易剩余~(Surplus)~为~$t_i-z_i(\mv{t}_{-i})$。
对卖者而言,投标人的期望交易剩余
\begin{equation*}
    E\left[T_i-z_i(\mv{T}_{-i})\right]
\end{equation*}
则是其要付出的信息租金~(Information Rent)。因为只有投标人~$i$~知道其真实估价,
故要让其报出真实估价,必须要为其提供一定的剩余,否则~$i$~将没有激励参与竞购。
这也说明在不完全信息下,卖者不可能通过最优机制榨取全部的交易剩余。


\section{效率机制}

直观上说,机制的效率关心的是物品是否分配到最能实现其价值的参与者手中,
即是否将物品分配给对其估计最高的投标人。


\subsection{机制的效率}

称分配规则~$\mv{\pi}^e:\mathcal{T}\rightarrow\mv{\Delta}$~是有效率的
~(简称为效率分配规则,Efficient,E),
若其最大化社会福利~(Social Welfare),形式上即对~$\forall\mv{t}
\in\mathcal{T}$均有
\begin{equation}
    \mv{\pi}^e(\mv{t})\in\mathop{\arg\max}\limits_{\mv{\pi}} 
         \sum_{i\in\mathcal{N}}\pi_it_i
\end{equation}
同时,称采用效率分配规则的机制为效率机制。给定效率分配规则~$\mv{\pi}^e$,定义
\begin{eqnarray}
    W(\mv{t}) &\triangleq& \sum_{i\in\mathcal{N}} \pi^e_i(\mv{t})t_i \\
    W_{-i}(\mv{t}) &\triangleq& \sum_{j\neq i} \pi^e_j(\mv{t})t_j
\end{eqnarray}
分别表示参与人类型为~$\mv{t}$~时的总社会福利和除参与人~$i$~外其他参与人的总福利。

\subsection{VCG~机制} 

由效率的定义可知机制的效率从本质上说取决于
其采用的分配规则,但效率分配规则通常并不唯一,同时并非所有的效率分配规则都能实现,
VCG (Vickrey-Clarke-Groves)~机制~$(\mv{\pi}^e, \mv{\mu}^{\mathrm{V}})$~给出了
效率分配规则的一种说实话实现,其定价规则取为某参与人的有效出现带给其他人的总福利损失~(外部性),
即对~$\forall i\in\mathcal{N}$~有
\begin{equation}
    \mu^{\mathrm{V}}_i \triangleq W(\underline{\theta}_i, \mv{t}_{-i}) - W_{-i}(\mv{t})
\end{equation}
注意到,在单物品拍卖的情景下,上述的~VCG~机制就是经典的二价~(SP)~机制。

\textbf{VCG~机制的激励兼容性}~ ~
考虑参与人~$i$,假设其余参与人均采用实话实说策略,则参与人~$i$~报价~$s_i$~的收益为
\begin{eqnarray*}
    \pi^e_i(s_i,\mv{t}_{-i})t_i - \mu_i^{\mathrm{V}}(s_i,\mv{t}_{-i}) 
         &=& \pi^e_i(s_i,\mv{t}_{-i})t_i + W_{-i}(s_i,\mv{t}_{-i}) - W(\underline{\theta}_i,\mv{t}_{-i}) \\
	 &=& \sum_{i\in\mathcal{N}} \pi^e_j(s_i,\mv{t}_{-i})t_j - W(\underline{\theta}_i,\mv{t}_{-i}) \\
	 &\le& W(\mv{t}) - W(\underline{\theta}_i,\mv{t}_{-i})
\end{eqnarray*}
注意到社会福利的定义,最后的不等号成立,由此可知~VCG~机制是激励兼容的。

\textbf{VCG~机制的个体理性}~ ~
由~VCG~机制的激励兼容性,根据上面的期望收益表达式,可得参与人~$i$~的均衡收益函数为
\begin{eqnarray}
    \nonumber
    U_i^\mathrm{V}(t_i) &=& \mathbb{E}[W(t_i,\mv{t}_{-i}) - W(\underline{\theta}_i,\mv{t}_{-i})] \\
                        &=& \mathbb{E}[W(t_i,\mv{t}_{-i})] - c_i^\mathrm{V} \label{VCG_payoff_equ}
\end{eqnarray}
且其为单调递增的凸函数,同时注意到~$U_i^\mathrm{V}(\underline{\theta}_i)=0$,
从而有~$U_i^\mathrm{V}(t_i)\ge0$,故~VCG~机制也是个体理性~IR~的。

\textbf{VCG~机制的收益特性}~ ~
在单物品拍卖的场景下,VCG~机制是所有满足~IC~和~IR~条件的效率机制中,
使得拍卖人的期望收益最大~(也即让参与人的期望花费最高)~的机制。

注意到,IC~条件使得收益等价原理可应用于这些机制,对任一个这样的机制,
参与人~$i$~的均衡收益函数~$U_i$~跟~$U_i^\mathrm{V}$~只差一个常数,记为~$U_i - U_i^\mathrm{V} = c_i$。
于是有~$U_i(\underline{\theta}_i) = c_i + U_i^\mathrm{V}(\underline{\theta}_i) = c_i$,
进一步由~IR~条件可知~$c_i\ge0$,从而参与人~$i$~的均衡收益将不低于在~VCG~机制下,
故在此设定下~VCG~机制最大化拍卖人的期望收益。

\textbf{VCG~机制的唯一性~(GLH~定理)}~ ~





\section{预算平衡} 

称机制~$\mathcal{D} = \{\mv{\pi}, \mv{\mu}\}$~能够平衡预算,若其对~$\forall \mv{t}\in\mv{\mathcal{T}}$~均有
\begin{equation}
    \sum_{i\in\mathcal{N}} \mu_i(\mv{t}) = 0
\end{equation}

\subsection{AGV~机制}

Arrow-d'Aspremont-G\'{e}rard-Varet (AGV)~机制~$(\pi^e, \mu^\mathrm{A})$~保证预算的平衡,
其分配规则采用效率分配规则,定价规则如下
\begin{equation}
    \mu_i^\mathrm{A}(\mv{t}) = \frac{1}{N-1} \sum_{j\neq i} \mathbb{E}_{\mv{t}_{-j}}[W_{-j}(t_j,\mv{t}_{-j})]
                             - \mathbb{E}_{\mv{t}_{-i}}[W_{-i}(t_i,\mv{t}_{-i})]
\end{equation}
注意到,在上述定价规则下,
\begin{eqnarray*}
    \sum_{i\in\mathcal{N}} \mu_i^\mathrm{A}(\mv{t}) &=& \frac{1}{N-1} \sum_{i\in\mathcal{N}}\sum_{j\neq i} \mathbb{E}_{\mv{t}_{-j}}[W_{-j}(t_j,\mv{t}_{-j})]
                             - \sum_{i\in\mathcal{N}} \mathbb{E}_{\mv{t}_{-i}}[W_{-i}(t_i,\mv{t}_{-i})] \\
			     &=& \frac{\mathrm{sum}}{N-1} \left(\left[
				 \begin{array}{cccc}
				     0 & \mathbb{E}[W_{-2}(t_2,\mv{t}_{-2})] & \cdots & \mathbb{E}[W_{-N}(t_N,\mv{t}_{-N})] \\
				     \mathbb{E}[W_{-1}(t_1,\mv{t}_{-1})] & 0 & \cdots & \mathbb{E}[W_{-N}(t_N,\mv{t}_{-N})] \\
				     \vdots & \vdots & \ddots & \vdots \\
				     \mathbb{E}[W_{-1}(t_1,\mv{t}_{-1})] & \mathbb{E}[W_{-2}(t_2,\mv{t}_{-2})] & \cdots & 0 
				 \end{array}
			         \right]\right) \\
			     & & - \sum_{i\in\mathcal{N}} \mathbb{E}_{\mv{t}_{-i}}[W_{-i}(t_i,\mv{t}_{-i})] \\
			     &=& 0
\end{eqnarray*}
式中,sum$(\cdot)$~为矩阵元素求和函数,由上述运算可知~AGV~机制是预算平衡的。
注意到,VCG~机制下的价格等于参与人给其他人造成的福利损失,而在~AGV~机制下,
价格的确定更加复杂,直观上价格等于其他人的平均期望福利与自己的期望福利之差。

\textbf{AGV~机制的激励兼容性}~ ~
考虑参与人~$i$,假设其余参与人均采用实话是说策略,则参与人~$i$~报价~$s_i$~的期望收益为
\begin{equation*}   
    \mathbb{E}_{\mv{t}_{-i}}[\pi^e_i(s_i,\mv{t}_{-i})t_i + W_{-i}(t_i,\mv{t}_{-i})]]
    - \mathbb{E}_{\mv{t}_{-i}}[\frac{1}{N-1} \sum_{j\neq i} \mathbb{E}_{\mv{t}_{-j}}[W_{-j}(t_j,\mv{t}_{-j})]]
\end{equation*}   
注意到上式中,第一个期望项在~$s_i = t_i$~时达到最大,第二项为常数,记为~$c_i^\mathrm{A}$,
故说实话也是参与人~$i$~的最佳策略。
可知~AGV~机制是激励兼容的,但一般不能保证其同时也是~IR~的。
同时可得到~AGV~机制下,参与人~$i$~的均衡收益函数为
\begin{equation}
    \label{AGV_payoff_equ}
    U_i^\mathrm{A}(t_i) = \mathbb{E}_{\mv{t}_{-i}}[W(t_i,\mv{t}_{-i})] - c_i^\mathrm{A}
\end{equation}   



\subsection{E + IC  + IR + BB}

一般不能保证~VCG~机制的预算平衡性和~AGV~机制的~IR~特性,但是可以联合~VCG~和~AGV~机制,
得到能同时满足效率、激励兼容、个体理性和预算平衡四个特性的机制。其前提是~VCG~机制有正的期望盈余。

这一要求是必要的。因为在所有满足~IC~和~IR~条件的效率机制中,VCG~机制的期望收益最高,
故若~VCG~不能平衡预算,则其他满足这三个条件的机制均不能平衡预算。充分性则可以通过构造出具体的机制来证明,
设~VCG~机制有期望的盈余,而~AGV~机制是预算平衡的,故有
\begin{equation*}
    \mathbb{E}\left[\sum_{i\in\mathcal{N}}\mu_i^\mathrm{V}(\mv{t})\right] \ge
    \mathbb{E}\left[\sum_{i\in\mathcal{N}}\mu_i^\mathrm{A}(\mv{t})\right] = 0
\end{equation*}   
注意到两种机制均采用效率分配规则,而上述不等式说明在期望支付上~VCG~机制又大于~AGV~机制,
故在均衡收益函数上,VCG~机制必须小于~AGV~机制。联合式~(\ref{VCG_payoff_equ})~和~(\ref{AGV_payoff_equ})~可得
\begin{equation*}
    \sum_i c_i^\mathrm{V} \ge \sum_i c_i^\mathrm{A}, ~ ~ \Rightarrow \sum_i \left(c_i^\mathrm{V} - c_i^\mathrm{A} \right) \ge 0,
    ~ ~ \Rightarrow \sum_{i>2} \left(c_i^\mathrm{V} - c_i^\mathrm{A} \right) \ge c_1^\mathrm{A} - c_1^\mathrm{V}
\end{equation*} 
据此构造机制~$\mathcal{D}^4=(\pi^e,\mu^\mathrm{B})$,其采用效率分配规则,定价规则为
\begin{equation}
    \mu_i^\mathrm{B}(\mv{t}) = \mu_i^\mathrm{A}(\mv{t}) + d_i
\end{equation}
式中,$d_i = c_i^\mathrm{A} - c_i^\mathrm{V}$,$\forall i>1$, 
$d_1 = -\sum_{i=2}^N d_i = \sum_{i>2} \left(c_i^\mathrm{V} - c_i^\mathrm{A} \right) \ge c_1^\mathrm{A} - c_1^\mathrm{V}$。

首先,机制~$\mathcal{D}^4$~显然是效率机制,因为其采用效率分配规则;
其次,$\mathcal{D}^4$~平衡预算,这由~$d_i$~的定义和~AGV~机制平衡预算可得;
第三,$\mathcal{D}^4$~的定价规则跟~AGV~只差常数,故其必然也是激励兼容的;
最后,$\mathcal{D}^4$~下的均衡收益函数也跟~AGV~机制下的均衡收益函数只差常数,
即对~$\forall i>1$~则有
\begin{equation*}
    U_i^\mathrm{B}(t_i) = U_i^\mathrm{A} + d_i = U_i^\mathrm{A} + c_i^\mathrm{A} - c_i^\mathrm{V} = U_i^\mathrm{V}(t_i)
\end{equation*}   
即是~VCG~机制的均衡收益函数。而对~$i=1$~则有
\begin{equation*}
    U_1^\mathrm{B}(t_1) = U_1^\mathrm{A} + d_1 \ge U_1^\mathrm{A} + c_1^\mathrm{A} - c_1^\mathrm{V} = U_1^\mathrm{V}(t_1)
\end{equation*}   
综上,可得机制~$\mathcal{D}^4$~也是个体理性~IR~的。






\section{总结} 

由前面讨论可知,除~VCG~机制外,其余机制均依赖与具体的应用环境,
特别是最优机制和~AGV~机制依赖于参与人的类型分布。下表总结了上述
各种机制的特性,由表可知机制一般只能具有~5~个特性中的~3~个。
注意到,能同时具有~E + IC + IR + BB~4~个特性的机制并不普遍存在,
其依赖于~VCG~机制在特定应用场景下是否存在期望盈余,
同时其设计实际上联合了~VCG~和~AGV~两种机制。另外,在所有满足~E + IC + IR~的机制中,
VCG~机制最大化拍卖人的期望收益,但这一最大期望收益不高于~OPT~机制下的期望收益,
因为~OPT~机制一般不是效率机制。

从设计实现上说,VCG~机制不依赖于参与人分布,只需按照实现的报价,根据最大化社会福利
的原则分配物品,并按指定的规则计算价格。对于~AGV~机制,给定参与人分布就能
按照既定的方法得到相应的机制。OPT~机制和同时具有四种特性的机制则不同,还需要参与人的分布
满足一定的条件,比如~OPT~机制要求的正规条件,同时具有四种特性的机制则要求~VCG~存在期望盈余。
注意,实际上在非正规条件下同样可以设计出~OPT~机制,因此也可以说给定参与人分布就能设计出~OPT~机制,
另外在一般的应用环境中,正规条件一般都是满足的。

\begin{table}[!h]
    \caption{各种机制及其特性}
    \vspace{.2cm}
    \centering 
    \begin{tabular}{c|ccccc}
	\hline
	    & E & IC & IR & BB & RM \\
	\hline
	OPT &            & \checkmark  & \checkmark  &              & \checkmark  \\
	VCG & \checkmark & \checkmark  & \checkmark  &              &    \\
	AGV & \checkmark & \checkmark  &             &  \checkmark  &    \\
	\hline
    \end{tabular}
\end{table}



\end{document}
