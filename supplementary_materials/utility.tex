\documentclass[a4paper,12pt]{article}

\usepackage[BoldFont,SlantFont,CJKchecksingle]{xeCJK}
\usepackage[top=1.25in,bottom=1.25in,left=1in,right=1in]{geometry}
\usepackage{amsmath,amsthm,amssymb,latexsym,color}
\usepackage[CJKbookmarks=true,
            bookmarksnumbered=true,
            bookmarksopen=true,
            colorlinks =true, %注释掉此项则交叉引用为彩色边框
            pdfborder=001,    %注释掉此项则交叉引用为彩色边框
            citecolor =blue,%
            linkcolor =blue,
            ]{hyperref}
\usepackage{layout}
\usepackage{enumerate}
\usepackage{graphicx}
\usepackage{subfigure}

\newcommand{\mv}[1]{\mbox{\boldmath$#1$}}         % 自定义矩阵向量 斜+黑
\newcommand{\id}{\mathrm{d}}                      % 积分测度改为正体

\setCJKmainfont[BoldFont=黑体]{宋体}
\setCJKsansfont{STKaiti}% 设置缺省中文字体

\parindent 2em   %段首缩进

\begin{document}

\title{个人选择理论}
\author{李林静 \\ 
   E-Mail: linjing.li@ia.ac.cn  \vspace{1cm} \\
 
 
        中国科学院自动化研究所 (CASIA)\\
        复杂系统管理与控制国家重点实验室 (SKLMCCS)\\
        互联网大数据与安全信息学研究中心 (iBasic)  \vspace{1cm} \\ 
 
 
        中国科学院大学人工智能学院 \vspace{5cm}
}

\date{Version: \today}



\maketitle



\newpage


\section{基本设定}

在微观经济学中,对个人的选择行为~(Choice
behavior)~进行建模,通常是假定存在一个选择集合~(Choice set, the set
of alternatives):
\begin{equation*}
    X=\{x,y,z,\cdots\}
\end{equation*}
从手段上看,存在以下两种不同的方法:
\begin{itemize}
  \item 基于偏好~(preference-based),假设个人对选择集~$X$~中的各选项存在偏好序关系。
  \item 基于选择~(choice-based),即直接对观测到的选择行为进行建模。
\end{itemize}

\section{偏好,Preference}

基于偏好的方法,通常假设个人存在理性的~(Rational)~偏好关系,同时可以用效用函数~(Utility function)~来表示
这个理性的偏好关系~(通常还需对选择集~$X$~作进一步的假设,在选择集~$X$~有限的情况下,
理性的偏好关系总可以由一个效用函数来表示)。

\subsection{偏好关系}

偏好关系~(Preference relation),$\succeq$,定义了选择集~$X$~上的一种比较关系。
$\forall x,y\in X$,$x\succeq y$~表示选项~$x$~至少跟选项~$y$~一样好。两种派生关系:
\begin{itemize}
  \item 严格偏好关系~(Strict preference relation),$\succ$,$\forall x,y\in X$,
        $x\succ y \Leftrightarrow x\succeq y$,$y\nsucceq x$。
  \item 无差异关系~(Indifference),$\sim$,$\forall x,y\in X$,
        $x\sim y \Leftrightarrow x\succeq y$,$y\succeq x$。
\end{itemize}
\textbf{称选择集~$X$~上的偏好关系~$\succeq$~为理性的,若其同时满足完全性和传递性}:
\begin{itemize}
  \item 完全性~(Completeness),$\forall x,y\in X$,$x\succeq y$~和~$y\succeq x$~至少一个成立。
  \item 传递性~(Transitivity),$\forall x,y,z\in X$,若~$x\succeq y$~且~$y\succeq z$,则~$x\succeq z$。
\end{itemize}
完全性和传递性都是对偏好关系施加的很强约束,现实中,个人的行为极有可能违背这两条假设。
相关讨论可以参考行为经济学相关文献。

\subsection{效用函数}

现代意义上的效用函数,一般是指序数效用~(Ordinal),即只有效用函数值的大小关系有意义。更一般的说,
序数特性是指在严格单调增变换~(Strictly increasing transformation)~中保持不变的性质。
在传统意义上的,基数效用~(Cardinal)~还跟具体的效用值相关。

称函数~$u: X\rightarrow\mathbb{R}$~为表示~$X$~上偏好关系~$\succeq$~的效用函数,若~$\forall x,y\in X$
\begin{equation*}
    x\succeq y~~~\Leftrightarrow~~~u(x)\geq u(y)
\end{equation*}
通常,表示同一个偏好关系的效用函数并不唯一。因为序数特性在增变换下保持不变,
故若~$u$~是一个效用函数,函数~$f: \mathbb{R}\rightarrow \mathbb{R}$~为严格单调增函数,
则~$f\circ u$~为效用函数,且跟效用函数~$u$~表示同一个偏好关系。

\textbf{偏好关系能够被效用函数表示的必要条件是其为理性偏好。}
在选择集有限的情况下,容易证明这就是充分必要条件~(构造一个效用函数)。

直观解释:假设~$u(\cdot)$~表示~$X$~上的偏好~$\succeq$。
首先,因为~$u(\cdot)$~在~$X$~上定义,故~$\forall x,y\in X$,$u(x)$~和~$u(y)$~有定义,
且为实数值。注意到,在实数集~$\mathbb{R}$~上,$u(x)\geq u(y)$~和~$u(y)\geq u(x)$~必有一个成立,
故有~$x\succeq y$~或~$y\succeq x$,即完全性成立。其次,$\forall x,y,z\in X$,
若~$x\succeq y$~且~$y\succeq z$,因为~$u(\cdot)$~表示~$\succeq$,故有~$u(x)\geq u(y)$~且~$u(y)\geq u(z)$。
而在实数集上,显然有~$u(x)\geq u(z)$,亦即~$x\succeq z$,故传递性成立。


\section{选择,Choice }

基于选择的方法最先由~Samuelson~提出,通过在选择集上定义选择结构~(Choice structure)~来描述观测
到的个体选择行为。利用这一结构,可以定义显示偏好关系~(revealed preference relation)。通常,
基于选择的方法要求的知识要少于基于偏好的方法,其只需要个人已观测到的选择行为。

\subsection{选择结构与显示偏好}

选择集~$X$~上的选择结构~$(\mathcal{B},C(\cdot))$:
\begin{itemize}
  \item 预算集族~$\mathcal{B}$,由~$X$~的子集构成,$\forall B\in
        \mathcal{B}$~称为预算集~(budget set),$\mathcal{B}$~不一定包含~$X$~的全部子集。
  \item 选择规则~(对应,correpondence)~$C(\cdot)$,Choice rule,$\forall B\in
        \mathcal{B}$,$C(B)\subset B$。即个体面临预算集~$B$~时,
        $C(B)$~表示其可能做出的选择。
\end{itemize}

给定选择集~$X$~上的选择结构~$(\mathcal{B},C(\cdot))$,
则可以诱导如下的显示偏好关系~(Revealed preference relation)~$\succeq^\star$:
\begin{equation*}
    x\succeq^\star y~~~\Leftrightarrow~~~\exists B\in\mathcal{B}, x,y\in B, x\in C(B)
\end{equation*}
即~$x$~显示出至少跟~$y$~一样好。称~$x$~显示出偏好于~$y$,
若~$\exists B\in\mathcal{B}, x,y\in B, x\in C(B)$,但~$y\notin C(B)$。


\subsection{显示偏好弱公理}

\textbf{称选择结构~$(\mathcal{B},C(\cdot))$~满足显示偏好弱公理~
(Weak axiom of revealed preference)~,若其具有如下特性}:
\begin{itemize}
  \item 对某(些)~$B\in\mathcal{B}$,若~$x,y\in B$~有~$x\in C(B)$;
  \item 则对~$\forall B^\prime\in\mathcal{B}$,且~$x,y\in B^\prime$,
        若~$y\in C(B^\prime)$,必有~$x\in C(B^\prime)$。
\end{itemize}
即当~$y$~存在时,$x$~被选择,则满足弱公理的选择结构不能存在包含~$x,y$~的预算集,
选择规则在其上包含~$y$~而不包含~$x$。利用显示偏好关系,
弱公理可以表述为:若~$x\succeq^\star y$~($x$~显示出至少跟~$y$~一样的好),
则~$y$~不能显示出偏好于~$x$。

弱公理也等价于以下表述,对~$B,B^\prime\in\mathcal{B}$,
$x,y\in B$,$x,y\in B^\prime$,若~$x\in C(B)$,$y\in C(B^\prime)$,
则必有~$\{x,y\}\subset C(B)$~和~$\{x,y\}\subset C(B^\prime)$。
进一步可得,$\forall B,B^\prime\in\mathcal{B}$,若~$C(B)\cap B^\prime\neq\emptyset$,
且~$C(B^\prime)\cap B\neq\emptyset$,则有~$C(B)\cap B^\prime\subset C(B^\prime)$~和~
$C(B^\prime)\cap B\subset C(B)$。若选择规则为单值,则有~$C(B^\prime)=C(B)$,
而且条件~$C(B)\cap B^\prime\neq\emptyset$,等价于~$C(B)\in B^\prime$。
~$C(B^\prime)\cap B\neq\emptyset$,等价于~$C(B^\prime)\in B$。


\section{偏好与选择间的关系}

对于上述两种方法,一个疑问是他们是否等价?通常而言,二者是不等价的,
基于选择的方法所要求的知识要少于基于偏好的方法。若选择结构的预算集族包含
选择集的全体子集,则二者等价。

首先考虑由偏好关系生成的~(generate)~选择结构。
设在选择集~$X$~上有定义好的偏好关系~$\succeq$~和预算集族~$\mathcal{B}$,
则可以由~$\succeq$~生成~$X$~上的一个选择结构,
记为~$(\mathcal{B},C^\star(\cdot,\succeq))$,其选择规则定义为:
\begin{equation*}
    \forall B\in\mathcal{B},~C^\star(B,\succeq)=\{x\in B \ | \ x\succeq y,~\forall y\in B\}
\end{equation*}

\textbf{选择集~$X$~上的偏好关系~$\succeq$~是理性的,
则其生成的选择结构~$(\mathcal{B},C^\star(\cdot,\succeq))$~满足弱公理}。

直观解释:若~$\exists B\in\mathcal{B}, x,y\in B, x\in C^\star(B,\succeq)$,
则按照生成规则的定义有~$x\succeq y$。现假设~$\exists B^\prime\in\mathcal{B},
x,y\in B^\prime, y\in C^\star(B^\prime,\succeq)$,为说明生成的选择结构满足弱公理,
还需要证明~$x\in C^\star(B^\prime,\succeq)$。
注意到,按生成规则可得~$\forall z\in B^\prime, y\succeq z$,前面已经得到~$x\succeq y$,
故由传递性~(理性偏好)~可得~$\forall z\in B^\prime, x\succeq z$,即~$x\in C^\star(B^\prime,\succeq)$。

另一方面,考虑选择结构的理性化。
给定选择集~$X$~上的选择结构~$(\mathcal{B},C(\cdot))$~和理性偏好~$\succeq$,
称~$\succeq$~是选择规则~$C(\cdot)$~相对于预算集族~$\mathcal{B}$~的理性化
~($\succeq$~在预算集族~$\mathcal{B}$~上理性化选择规则~$C(\cdot)$),
若~$\succeq$~生成选择结构~$(\mathcal{B},C(\cdot))$,即
\begin{equation*}
    \forall B\in\mathcal{B},~C(B)=C^\star(B,\succeq)
\end{equation*}
因为理性偏好生成的选择结构满足弱公理,故一个选择结构可理性化的必要条件就是其必须满足弱公理。
但这并不充分,选择结构能否理性化还依赖于其预算集族。

\textbf{若选择集~$X$~上的选择结构~$(\mathcal{B},C(\cdot))$~满足弱公理,
且~$\mathcal{B}$~包含~$X$~的所有不超过三个元素的子集,
则其可被唯一的理性偏好关系理性化,且这一偏好关系就是选择结构诱导的显示偏好关系~$\succeq^\star$}。

直观解释:首先说明唯一性,因为~$\mathcal{B}$~包含所有的二元子集,
故理性化偏好若存在的话必定唯一,否则将会跟~$C(\cdot)$~在二元子集上的行为矛盾。

其次,说明显示偏好~$\succeq$~生成选择结构~$(\mathcal{B},C(\cdot))$,利用证明集合相等的方法。
$\forall B\in\mathcal{B}$,若~$x\in C(B)$,则~$\forall y\in B$,根据显示偏好关系的定义,
可得~$x\succeq^\star y$。根据生成规则,可得~$x\in C^\star(B,\succeq^\star)$,
从而有~$C(B)\subset C^\star(B,\succeq^\star)$。若有~$x\in C^\star(B,\succeq^\star)$,
则根据生成规则,可得~$\forall y\in B, x\succeq^\star y$,从而~$x\in C(B)$,
即~$C^\star(B,\succeq^\star)\subset C(B)$。

最后,说明显示偏好关系~$\succeq$~是理性的。因为~$\mathcal{B}$~包含所有的二元子集,
故~$\forall \{x,y\}\in\mathcal{B}$,要么~$x\in C(\{x,y\})$,要么~$y\in C(\{x,y\})$,
或者~$x,y\in C(\{x,y\})$。所以,要么~$x\succeq^\star y$,要么~$y\succeq^\star x$,
或者~$x\sim y$,即显示偏好关系满足完全性,以下说明其满足传递性。
假设有~$x\succeq^\star y$,$y\succeq^\star z$,需要证明~$x\succeq^\star z$。
注意到预算集族包含所有三元子集,故~$\{x,y,z\}\in\mathcal{B}$。
若~$x\in C(\{x,y,z\})$,则有~$x\succeq^\star z$。若~$y\in C(\{x,y,z\})$,
因为~$x\succeq^\star y$,弱公理保证~$x\in C(\{x,y,z\})$,从而~$x\succeq^\star z$。
若~$z\in C(\{x,y,z\})$,
因为~$y\succeq^\star z$,弱公理保证~$y\in C(\{x,y,z\})$,
从而由上面的推理同样可得~$x\succeq^\star z$。


\end{document}
